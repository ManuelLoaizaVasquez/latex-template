\title{C\'odigo Nombre del Curso}
\author{Tu nombre aqu\'i}
\date{Fecha}

\begin{document}

\maketitle

\vspace*{-0.25in}
\centerline{Tu universidad aqu\'i}
\centerline{Tu regi\'on y pa\'is aqu\'i}
\centerline{\mailto{tucorreoaqui@pucp.edu.pe}}
\vspace*{0.15in}

\begin{framed}
  Esta es una plantilla en \LaTeX.
  Util\'izala en tus listas de ejercicios, notas de clases y mucho m\'as.
\end{framed}

\begin{statement}{1}
    Este es el enunciado de un problema.
    Colocar el enunciado de un problema de un color diferente permite distinguir entre el problema y la soluci\'on.
\end{statement}

\begin{proof}
    Tipea tus soluciones en esta secci\'on. Usa las definiciones, lemas y ejemplos cuando sean necesarios.
    \begin{defn}
        Definimos $\exp(x)$ para $x \in \BR$ el valor de $$\sum_{i = 0}^\infty\frac{x^i}{i!}.$$
    \end{defn}
    Al igual que en la definici\'on anterior,
    usar ecuaciones en l\'ineas separadas cuando sea posible,
    esto hace tu problema m\'as le\'ible. Usa \texttt{align*} para lista de igualdades:
    \begin{align*}
        0 &= 0 + 0 + 0 + 0 + \dots\\
        &= (1 - 1) + (1 - 1) + \dots \\
        &= 1 + (-1 + 1) + (-1 + 1) + \dots \\
        &= 1 + 0 + 0 + 0 \dots \\
        &= 1.
    \end{align*}
    Si necesitas listar cosas, usa \texttt{enumerate} o \texttt{itemize}.
    Por ejemplo:
    \begin{enumerate}
        \item Debo ser aceptado en Google antes de egresar.
        \item Debo clasificar al ACM ICPC World Finals.
        \item Debo terminar dentro del top $10$ en el mundial IEEEXtreme.
        \item Debo trabajar como ingeniero de software remoto en paralelo a la universidad.
        \item Debo entrenar al menos $20$ horas semanales Theoretical Computer Science en paralelo a la universidad.
        \item Debo crear mi lenguaje de programaci\'on orientado a objetos antes de egresar.
    \end{enumerate}
    Todas las secciones \texttt{proof} terminan con el cuadrado que simboliza el hecho de que ha concluido la demostraci\'on.
\end{proof}

\begin{statement}{2}
    Si la soluci\'on de un problema es f\'acil de chequear si es correcta, ¿el problema debe ser f\'acil de resolver?
\end{statement}

\begin{proof}
    Buena suerte.
\end{proof}

\end{document}